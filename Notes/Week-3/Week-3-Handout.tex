\documentclass[]{article}
\usepackage[T1]{fontenc}
\usepackage{lmodern}
\usepackage{amssymb,amsmath}
\usepackage{ifxetex,ifluatex}
\usepackage{fixltx2e} % provides \textsubscript
% use upquote if available, for straight quotes in verbatim environments
\IfFileExists{upquote.sty}{\usepackage{upquote}}{}
\ifnum 0\ifxetex 1\fi\ifluatex 1\fi=0 % if pdftex
  \usepackage[utf8]{inputenc}
\else % if luatex or xelatex
  \ifxetex
    \usepackage{mathspec}
    \usepackage{xltxtra,xunicode}
  \else
    \usepackage{fontspec}
  \fi
  \defaultfontfeatures{Mapping=tex-text,Scale=MatchLowercase}
  \newcommand{\euro}{€}
\fi
% use microtype if available
\IfFileExists{microtype.sty}{\usepackage{microtype}}{}
\usepackage[margin=1in]{geometry}
\usepackage{color}
\usepackage{fancyvrb}
\newcommand{\VerbBar}{|}
\newcommand{\VERB}{\Verb[commandchars=\\\{\}]}
\DefineVerbatimEnvironment{Highlighting}{Verbatim}{commandchars=\\\{\}}
% Add ',fontsize=\small' for more characters per line
\usepackage{framed}
\definecolor{shadecolor}{RGB}{248,248,248}
\newenvironment{Shaded}{\begin{snugshade}}{\end{snugshade}}
\newcommand{\KeywordTok}[1]{\textcolor[rgb]{0.13,0.29,0.53}{\textbf{{#1}}}}
\newcommand{\DataTypeTok}[1]{\textcolor[rgb]{0.13,0.29,0.53}{{#1}}}
\newcommand{\DecValTok}[1]{\textcolor[rgb]{0.00,0.00,0.81}{{#1}}}
\newcommand{\BaseNTok}[1]{\textcolor[rgb]{0.00,0.00,0.81}{{#1}}}
\newcommand{\FloatTok}[1]{\textcolor[rgb]{0.00,0.00,0.81}{{#1}}}
\newcommand{\CharTok}[1]{\textcolor[rgb]{0.31,0.60,0.02}{{#1}}}
\newcommand{\StringTok}[1]{\textcolor[rgb]{0.31,0.60,0.02}{{#1}}}
\newcommand{\CommentTok}[1]{\textcolor[rgb]{0.56,0.35,0.01}{\textit{{#1}}}}
\newcommand{\OtherTok}[1]{\textcolor[rgb]{0.56,0.35,0.01}{{#1}}}
\newcommand{\AlertTok}[1]{\textcolor[rgb]{0.94,0.16,0.16}{{#1}}}
\newcommand{\FunctionTok}[1]{\textcolor[rgb]{0.00,0.00,0.00}{{#1}}}
\newcommand{\RegionMarkerTok}[1]{{#1}}
\newcommand{\ErrorTok}[1]{\textbf{{#1}}}
\newcommand{\NormalTok}[1]{{#1}}
\usepackage{graphicx}
% Redefine \includegraphics so that, unless explicit options are
% given, the image width will not exceed the width of the page.
% Images get their normal width if they fit onto the page, but
% are scaled down if they would overflow the margins.
\makeatletter
\def\ScaleIfNeeded{%
  \ifdim\Gin@nat@width>\linewidth
    \linewidth
  \else
    \Gin@nat@width
  \fi
}
\makeatother
\let\Oldincludegraphics\includegraphics
{%
 \catcode`\@=11\relax%
 \gdef\includegraphics{\@ifnextchar[{\Oldincludegraphics}{\Oldincludegraphics[width=\ScaleIfNeeded]}}%
}%
\ifxetex
  \usepackage[setpagesize=false, % page size defined by xetex
              unicode=false, % unicode breaks when used with xetex
              xetex]{hyperref}
\else
  \usepackage[unicode=true]{hyperref}
\fi
\hypersetup{breaklinks=true,
            bookmarks=true,
            pdfauthor={},
            pdftitle={DCF Week 3 Activity},
            colorlinks=true,
            citecolor=blue,
            urlcolor=blue,
            linkcolor=magenta,
            pdfborder={0 0 0}}
\urlstyle{same}  % don't use monospace font for urls
\setlength{\parindent}{0pt}
\setlength{\parskip}{6pt plus 2pt minus 1pt}
\setlength{\emergencystretch}{3em}  % prevent overfull lines
\setcounter{secnumdepth}{0}

%%% Change title format to be more compact
\usepackage{titling}
\setlength{\droptitle}{-2em}
  \title{DCF Week 3 Activity}
  \pretitle{\vspace{\droptitle}\centering\huge}
  \posttitle{\par}
  \author{}
  \preauthor{}\postauthor{}
  \predate{\centering\large\emph}
  \postdate{\par}
  \date{Data and Computing Fundamentals}




\begin{document}

\maketitle


\section{Popular Names}\label{popular-names}

The relative popularity of different names for babies varies over the
years and decades. In this exercise, you're going to visualize how the
popularity of names varies in time.

\subsection{Tasks}\label{tasks}

You're going to be working individually first, and then in groups.

\begin{itemize}
\itemsep1pt\parskip0pt\parsep0pt
\item
  First, undertake an \emph{analysis} of data transfiguration tasks.
  What kind of data table is the final goal, what is available as the
  input data. Find a path from input to the final goal in terms of data
  verbs.
\item
  Second, \emph{design} the steps of the transfiguration process.
  Describe these in English (if only because your English is a lot
  better than your R at this stage in your education).
\item
  Third, \emph{implement} the design in R.
\end{itemize}

In any work, it's important to have a goal and a plan for data table
transfigurations. This need not be elaborate.

\subsection{Objective}\label{objective}

Create a graph like the following using name of interest to you.

\includegraphics{./Week-3-Handout_files/figure-latex/unnamed-chunk-4.pdf}

The raw material you have is the \texttt{BabyNames} data set in the DCF
package. Write a few example lines in the format of the
\texttt{BabyNames}, identifying the variable names and typical levels or
values.

\vspace*{1.5in}

Point out which variables are categorical. These can potentially be used
for defining groups of cases.

\subsubsection{First, Individually \ldots{}}\label{first-individually}

\paragraph{Step 1.}\label{step-1.}

Analyze the graphic to figure out what a glyph-ready data table should
look like. Mostly, this involves figuring out what variables are
represented in the graph. Write down a small example of a glyph-ready
data frame that you think could be used to make something in the form of
the graphic.

\vspace*{1.5in}

\begin{itemize}
\itemsep1pt\parskip0pt\parsep0pt
\item
  What variable(s) from the raw data table do not appear at all in the
  graph?
\item
  What variable(s) in the graph are similar to corresponding variables
  in the raw data table, but might have been transformed in some way.
\end{itemize}

\paragraph{Step 2}\label{step-2}

Consider how the cases differ between the raw input and the glyph-ready
table.

\begin{itemize}
\itemsep1pt\parskip0pt\parsep0pt
\item
  Have cases been \textbf{filtered} out?
\item
  Have cases been grouped and \textbf{aggregated/summarized} within
  groups in any way?
\item
  Have any new variables been introduced?
\end{itemize}

\paragraph{Step 3}\label{step-3}

Using English, write down a sequence of steps that will accomplish the
transfiguration from the raw data table to your hypothesized glyph-ready
data table.

\paragraph{Step 4: Confer with your
colleagues}\label{step-4-confer-with-your-colleagues}

As a group, compare your different analyses in Steps 1 through 3. Your
goal is to develop a consensus for the design in Step 3.

\begin{itemize}
\item
  \textbf{FIRST}, each person in turn should present his or her analysis
  with others in the group \textbf{listening}. Don't be shy about saying
  things like, ``But I'm not sure about this,'' or, ``Hearing what
  person X just said, I realize that this is wrong. I'll explain how
  it's wrong.''
\item
  \textbf{AFTER} everyone has presented, start a discussion to develop a
  consensus about the sequence of operations and the parameters of each
  (e.g.: What's being filtered out.) Use the whiteboard to keep a record
  of your consensus.
\end{itemize}

\paragraph{Step 5: First individually, then as a
group.}\label{step-5-first-individually-then-as-a-group.}

Translate your design, step by step, into R.

\paragraph{Step 6: Implementation}\label{step-6-implementation}

Now you can start writing the commands themselves. Do so, try to
identify and solve any problems that arise, and make your glyph-ready
data.

For graphing, you can use this template:

\begin{Shaded}
\begin{Highlighting}[]
\NormalTok{Results %>%}\StringTok{ }\KeywordTok{ggplot}\NormalTok{(}\KeywordTok{aes}\NormalTok{(}\DataTypeTok{x=}\NormalTok{year,}\DataTypeTok{y=}\NormalTok{total,}\DataTypeTok{group=}\NormalTok{name)) +}
\StringTok{   }\KeywordTok{geom_line}\NormalTok{( }\DataTypeTok{size=}\DecValTok{3}\NormalTok{, }\DataTypeTok{alpha=}\NormalTok{.}\DecValTok{5}\NormalTok{, }\KeywordTok{aes}\NormalTok{(}\DataTypeTok{color=}\NormalTok{name)) +}
\StringTok{   }\KeywordTok{ylab}\NormalTok{(}\StringTok{"Popularity"}\NormalTok{) +}\StringTok{ }\KeywordTok{xlab}\NormalTok{(}\StringTok{"Year"}\NormalTok{)}
\end{Highlighting}
\end{Shaded}

\paragraph{Finally \ldots{}}\label{finally}

A list of Bible-related names is available this way:

\begin{Shaded}
\begin{Highlighting}[]
\NormalTok{BibleNames <-}\StringTok{ }\KeywordTok{fetchData}\NormalTok{( }\StringTok{"DCF/BibleNames.csv"} \NormalTok{)}
\end{Highlighting}
\end{Shaded}

\begin{verbatim}
## Retrieving from http://www.mosaic-web.org/go/datasets/DCF/BibleNames.csv
\end{verbatim}

Using this data table, and working as a group:

\begin{itemize}
\itemsep1pt\parskip0pt\parsep0pt
\item
  Make a data table showing the most popular biblenames over all the
  years.
\item
  Make an informative plot showing the trends over the years of
  Bible-related names as a proportion of all names.
\end{itemize}

\end{document}
