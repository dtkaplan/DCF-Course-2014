\documentclass[]{article}
\usepackage[T1]{fontenc}
\usepackage{lmodern}
\usepackage{amssymb,amsmath}
\usepackage{ifxetex,ifluatex}
\usepackage{fixltx2e} % provides \textsubscript
% use upquote if available, for straight quotes in verbatim environments
\IfFileExists{upquote.sty}{\usepackage{upquote}}{}
\ifnum 0\ifxetex 1\fi\ifluatex 1\fi=0 % if pdftex
  \usepackage[utf8]{inputenc}
\else % if luatex or xelatex
  \ifxetex
    \usepackage{mathspec}
    \usepackage{xltxtra,xunicode}
  \else
    \usepackage{fontspec}
  \fi
  \defaultfontfeatures{Mapping=tex-text,Scale=MatchLowercase}
  \newcommand{\euro}{€}
\fi
% use microtype if available
\IfFileExists{microtype.sty}{\usepackage{microtype}}{}
\usepackage[margin=1in]{geometry}
\usepackage{color}
\usepackage{fancyvrb}
\newcommand{\VerbBar}{|}
\newcommand{\VERB}{\Verb[commandchars=\\\{\}]}
\DefineVerbatimEnvironment{Highlighting}{Verbatim}{commandchars=\\\{\}}
% Add ',fontsize=\small' for more characters per line
\usepackage{framed}
\definecolor{shadecolor}{RGB}{248,248,248}
\newenvironment{Shaded}{\begin{snugshade}}{\end{snugshade}}
\newcommand{\KeywordTok}[1]{\textcolor[rgb]{0.13,0.29,0.53}{\textbf{{#1}}}}
\newcommand{\DataTypeTok}[1]{\textcolor[rgb]{0.13,0.29,0.53}{{#1}}}
\newcommand{\DecValTok}[1]{\textcolor[rgb]{0.00,0.00,0.81}{{#1}}}
\newcommand{\BaseNTok}[1]{\textcolor[rgb]{0.00,0.00,0.81}{{#1}}}
\newcommand{\FloatTok}[1]{\textcolor[rgb]{0.00,0.00,0.81}{{#1}}}
\newcommand{\CharTok}[1]{\textcolor[rgb]{0.31,0.60,0.02}{{#1}}}
\newcommand{\StringTok}[1]{\textcolor[rgb]{0.31,0.60,0.02}{{#1}}}
\newcommand{\CommentTok}[1]{\textcolor[rgb]{0.56,0.35,0.01}{\textit{{#1}}}}
\newcommand{\OtherTok}[1]{\textcolor[rgb]{0.56,0.35,0.01}{{#1}}}
\newcommand{\AlertTok}[1]{\textcolor[rgb]{0.94,0.16,0.16}{{#1}}}
\newcommand{\FunctionTok}[1]{\textcolor[rgb]{0.00,0.00,0.00}{{#1}}}
\newcommand{\RegionMarkerTok}[1]{{#1}}
\newcommand{\ErrorTok}[1]{\textbf{{#1}}}
\newcommand{\NormalTok}[1]{{#1}}
\ifxetex
  \usepackage[setpagesize=false, % page size defined by xetex
              unicode=false, % unicode breaks when used with xetex
              xetex]{hyperref}
\else
  \usepackage[unicode=true]{hyperref}
\fi
\hypersetup{breaklinks=true,
            bookmarks=true,
            pdfauthor={},
            pdftitle={Choosing Data Verbs},
            colorlinks=true,
            citecolor=blue,
            urlcolor=blue,
            linkcolor=magenta,
            pdfborder={0 0 0}}
\urlstyle{same}  % don't use monospace font for urls
\setlength{\parindent}{0pt}
\setlength{\parskip}{6pt plus 2pt minus 1pt}
\setlength{\emergencystretch}{3em}  % prevent overfull lines
\setcounter{secnumdepth}{0}

%%% Change title format to be more compact
\usepackage{titling}
\setlength{\droptitle}{-2em}
  \title{Choosing Data Verbs}
  \pretitle{\vspace{\droptitle}\centering\huge}
  \posttitle{\par}
  \author{}
  \preauthor{}\postauthor{}
  \date{}
  \predate{}\postdate{}




\begin{document}

\maketitle


\subsection{What's your input and your intended
output?}\label{whats-your-input-and-your-intended-output}

Questions about your input and output data frames:

\begin{itemize}
\itemsep1pt\parskip0pt\parsep0pt
\item
  For the input data table(s):

  \begin{enumerate}
  \def\labelenumi{\arabic{enumi}.}
  \itemsep1pt\parskip0pt\parsep0pt
  \item
    What is the meaning of a case?
  \item
    How many cases are there?
  \item
    Is there more than one input data table?
  \end{enumerate}
\item
  For the output data table:

  \begin{enumerate}
  \def\labelenumi{\arabic{enumi}.}
  \itemsep1pt\parskip0pt\parsep0pt
  \item
    What is the meaning of a case?
  \item
    How many cases are there?
  \item
    Are there more cases in the input than the output?
  \end{enumerate}
\end{itemize}

\subsection{Choosing a Data Verb}\label{choosing-a-data-verb}

Walk through these questions. Stop when you hit the first ``yes''
answer.

\begin{enumerate}
\def\labelenumi{\arabic{enumi}.}
\item
  Are there two input tables? Use \texttt{join()}

  Otherwise \ldots{}
\item
  Is the ``physical'' meaning of a case different from input to output?
  Example: cases are people in input, but cases are age groups in the
  output. Use \texttt{group\_by()} and \texttt{summarise()}

  Otherwise \ldots{}
\item
  Are there fewer cases in the output than in the input? Use
  \texttt{filter()} or \texttt{head()}, \texttt{sample\_n()}, etc.

  \begin{enumerate}
  \def\labelenumii{\arabic{enumii}.}
  \item
    Does the output have the first few cases in the input? Use
    \texttt{head()} (or \texttt{tail()} for the last few cases).

    Otherwise \ldots{}
  \item
    Are the cases in the output randomly selected from the input? Use
    \texttt{sample\_n()}

    Otherwise \ldots{}
  \item
    Are the cases in the output a subset from the input of cases that
    meet one or more criteria? Use \texttt{filter()}
  \end{enumerate}

  Otherwise \ldots{}
\item
  There is a variable in the output that is not in the input? Example,
  there's population and area in the input. The output includes
  population density. Use \texttt{mutate()}

  Otherwise \ldots{}
\item
  Are there some variables in the input that are not in the output? Use
  \texttt{select()}

  Otherwise \ldots{}
\item
  Are the cases are the same in the input and the output, but in a
  different order. Use \texttt{arrange()}

  Otherwise
\item
  It sounds like the input is the same as the output. If so, you don't
  need any data verb. If you don't think this is compelling, Start over
  and reconsider the above questions.
\end{enumerate}

\subsection{Adverbs}\label{adverbs}

The arguments to a data verb include a data table and one or more
``adverbs'' that describe how the verb is to operate.

\subsubsection{Join}\label{join}

There are two data-verb arguments. No adverbs are needed, unless you
want to specify what cases to keep. See \texttt{merge()} in that
situation.

\subsubsection{Arrange}\label{arrange}

The quantity or quantities by which to arrange the cases.

\begin{Shaded}
\begin{Highlighting}[]
\NormalTok{HappinessIndex %>%}\StringTok{ }\KeywordTok{arrange}\NormalTok{( gdpPerCapita )}
\NormalTok{HappinessIndex %>%}\StringTok{ }\KeywordTok{arrange}\NormalTok{( region, gdpPerCapita )}
\CommentTok{# Sort by region, within each region break ties by GDP.}

\CommentTok{# Descending order, pipe the quantity to `desc()`}
\NormalTok{HappinessIndex %>%}\StringTok{ }\KeywordTok{arrange}\NormalTok{( region, gdpPerCapita %>%}\StringTok{ }\KeywordTok{desc}\NormalTok{() )}
\end{Highlighting}
\end{Shaded}

\end{document}
